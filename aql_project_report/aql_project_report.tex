\documentclass{article}

% if you need to pass options to natbib, use, e.g.:
%     \PassOptionsToPackage{numbers, compress}{natbib}
% before loading neurips_2019

% ready for submission
% \usepackage{neurips_2019}

% to compile a preprint version, e.g., for submission to arXiv, add add the
% [preprint] option:
%     \usepackage[preprint]{neurips_2019}

% to compile a camera-ready version, add the [final] option, e.g.:
\usepackage[schoolpaper]{neurips_2019}

% to avoid loading the natbib package, add option nonatbib:
%     \usepackage[nonatbib]{neurips_2019}

\usepackage[utf8]{inputenc} % allow utf-8 input
\usepackage[T1]{fontenc}    % use 8-bit T1 fonts
\usepackage{hyperref}       % hyperlinks
\usepackage{url}            % simple URL typesetting
\usepackage{booktabs}       % professional-quality tables
\usepackage{amsfonts}       % blackboard math symbols
\usepackage{nicefrac}       % compact symbols for 1/2, etc.
\usepackage{microtype}      % microtypography
\usepackage{graphicx}

\graphicspath{{images/}} 

\title{AQL Project Report}

% The \author macro works with any number of authors. There are two commands
% used to separate the names and addresses of multiple authors: \And and \AND.
%
% Using \And between authors leaves it to LaTeX to determine where to break the
% lines. Using \AND forces a line break at that point. So, if LaTeX puts 3 of 4
% authors names on the first line, and the last on the second line, try using
% \AND instead of \And before the third author name.

\author{%
  Frank Acquaye\thanks{http://acquayefrank.github.io} \\
  Faculty of Computer Science\\
  National Research University Higher School of Economics\\
  \texttt{fakvey@edu.hse.ru} \\
  % examples of more authors
  % \And
  % Coauthor \\
  % Affiliation \\
  % Address \\
  % \texttt{email} \\
  % \AND
  % Coauthor \\
  % Affiliation \\
  % Address \\
  % \texttt{email} \\
  % \And
  % Coauthor \\
  % Affiliation \\
  % Address \\
  % \texttt{email} \\
  % \And
  % Coauthor \\
  % Affiliation \\
  % Address \\
  % \texttt{email} \\
}

\begin{document}

\maketitle

\begin{abstract}
	In this paper we present a variant of the \textbf{Green Vehicle Routing Problem (GVRP)} with capacitated vehicles. In this variant of the GVRP, we present a situation in which distributors are faced with the daunting task of  managing a fleet of both traditional fuelled vehicles and alternate fuelled vehicles all this while maximizing profitability and minimizing cost. We propose a solution which uses \textbf{branch and bound} to maximize distribution path when needed and minimize distribution path when needed.
 \end{abstract}

\section{Introduction}
More and more distributors are faced with the daunting task of moving towards green energy in all aspects of operations. This often involves switching distribution vehicles to vehicles that consume alternate fuel. The challenge with alternate fuel however is the limited travel distance, before the need for refueling. Hence these distributors often have vehicles that use traditional fuel and vehicles that use alternate fuel. Our goal is therefore to produce an optimisation algorithm that maximizes the distribution paths of green fueled vehicles and minimizing the distribution paths of traditional vehicles all this while ensuring that  the total demand for services/products is met.  

\section{Problem Statement}
\end{document}
